\documentclass[a4paper, 12pt]{extreport}


%%  Русификация
\usepackage[utf8]{inputenc}
\usepackage[english, russian]{babel}


%% Греческий
%\usepackage{textgreek}

%% Математическая символика
\usepackage{mathtools}

%% Геометрия листа
\usepackage{vmargin}
\setpapersize{A4}

%% Поля страницы
\setmarginsrb{2cm}{1.5cm}{2cm}{1.5cm}{0pt}{0mm}{0pt}{13mm}

%% Защита от наползаний на поля
\sloppy

\begin{document}

\chapter{Множества}

\section{Введение в множества}

Основным понятием данной книги является \itshape множество\normalfont{}. Оно, по крайней мере, поверхностно, чрезвычайно просто. Множество -- это некий набор, группа, совокупность. Так, мы можем рассматривать множество всех студентов Московского физико-технического института в январе 2022 года, множество всех натуральных чисел, множество всех точек плоскости $\alpha$, удаленных на расстояние 5 сантиметров от заданной точки $P$, множество всех розовых слонов. \par
Множества не являются объектами реального мира, как столы и звёзды; они были созданы нашим мозгом, а не нашими руками. Ведро картошки не является множеством картофеля, множество молекул в капле воды не является тем же объектом, что и эта капля воды. Человеческий мозг может абстрагироваться, думать о разновидности различных объектов, которые связаны друг с другом каким-то общим свойством, и таким образом формировать множество объектов, обладающих данным свойством. Свойством в этом описании может быть даже  Так, хотя для множества, состоящего из чисел $2$, $7$, $12$, $13$, $29$, $34$ и $11,000$, вряд ли можно представить что-либо, связывающее именно эти числа, они связаны тем фактом, что мы мысленно объединили их. Георг Кантор, немецкий математик, основатель теории множеств, в серии своих публикаций последнего тридцатилетия девятнадцатого века выразил это примерно следующим образом: "Множество -- это объединение определённых и чётко оговоренных объектов нашего воображения; эти объекты называются элементами (членами) множества". \par
Объекты, из которых состоит множество, называют элементами или членами этого множества. Также говорят, что они \itshape принадлежат\normalfont{} этому множеству. \par
В этой книге мы хотим рассмотреть теорию множеств как основание для других математических дисциплин. Поэтому нас не интересуют множества людей или молекул. Мы будем иметь дело только с множествами \itshape математических\normalfont{} объектов. Такими объектами могут быть числа, точки пространства или множества. Вообще, первые три понятия могут быть определены в теории множеств как множества с конкретными свойствами, что мы и сделаем в следующих главах. Итак, с этого момента, из всех объектов нас интересуют только множества. Для наглядности мы говорим о множествах чисел или точек даже до того, как эти понятия строго определены. Однако, мы делаем это только в рамках примеров, упражнений и задач, а не в самой теории. Приведём примеры множеств математических объектов: \par

\end{document}

