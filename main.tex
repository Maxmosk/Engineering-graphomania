\documentclass[a5paper, 10pt]{extreport}


%%  Русификация
\usepackage[utf8]{inputenc}
\usepackage[english, russian]{babel}


%% Греческий
%\usepackage{textgreek}

%% Математическая символика
\usepackage{mathtools}

%% Геометрия листа
\usepackage{vmargin}
\setpapersize{A5}

%% Поля страницы
\setmarginsrb{2cm}{2cm}{1cm}{1.5cm}{0pt}{0mm}{0pt}{13mm}

%% Защита от наползаний на поля
\sloppy

\begin{document}

\chapter{Множества}

\section{Введение в множества}

Основным понятием данной книги является \itshape множество\normalfont{}. Оно, по крайней мере, поверхностно, чрезвычайно просто. Множество -- это некий набор, группа, совокупность. Так, мы можем рассматривать множество всех студентов Московского физико-технического института в январе 2022 года, множество всех натуральных чисел, множество всех точек плоскости $\alpha$, удаленных на расстояние 5 сантиметров от заданной точки $P$, множество всех розовых слонов. \par
Множества не являются объектами реального мира, как столы и звёзды; они были созданы нашим мозгом, а не нашими руками. Ведро картошки не является множеством картофеля, множество молекул в капле воды не является тем же объектом, что и эта капля воды. Человеческий мозг может абстрагироваться, думать о разновидности различных объектов, которые связаны друг с другом каким-то общим свойством, и таким образом формировать множество объектов, обладающих данным свойством. Свойства в этом описании могут быть чем-то не большим, чем способность подумать об объединении этих объектов. Так, хотя для множества, состоящего из чисел 2, 7, 12, 13, 29, 34 и 11,000, вряд ли можно представить что-либо, связывающее именно эти числа, они связаны тем фактом, что мы мысленно объединили их. Георг Кантор, немецкий математик, основатель теории множеств, в серии своих публикаций последнего тридцатилетия девятнадцатого века выразил это примерно следующим образом: "Множество есть объединение определённых чётко оговоренных объектов нашего воображения; эти объекты называются элементами (членами) множества". \par
Объекты, их которых состоит множество, называются элементами или членами этого множества. Мы также говорим, что они \itshape принадлежат\normalfont{} этому множеству. \par

\end{document}

